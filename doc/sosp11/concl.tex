\section{Conclusion}
\label{s:concl}

We presented \name{}, a system that provides practical and provable
confidentiality in the face of two significant threats---curious DBAs
and arbitrary compromises of the application server and
DBMS---confronting database-backed applications.  \name{} meets its
goals using three ideas: running queries efficiently over encrypted
data using a novel SQL-aware encryption strategy, dynamically
adjusting the encryption level using onions of encryption to minimize
the information revealed to the untrusted DBMS server, and chaining
encryption keys to user passwords in a way that allows only authorized
users to gain access to the same encrypted data.  The developer effort
required to express confidentiality policies for multi-user
applications is small, ranging between $11$ and $13$ unique
annotations of the application's database schema across three existing
applications (phpBB, HotCRP, and grad-apply).  The throughput penalty
of \name{} is modest, about $\tput$ for the TPC-C benchmark and $13\%$
for our synthetic phpBB workload.  Our results suggest that \name{} is
useful for applications where the ability to run confidentially over
an untrusted or outsourced infrastructure trumps achieving the highest
possible performance.

The source code for CryptDB is available for download at
\url{http://css.csail.mit.edu/cryptdb/}.

% This paper presented \name, a practical and novel system for ensuring
% data privacy on an untrusted SQL DBMS server.  \name uses three novel
% ideas to achieve its goal: an {\em SQL-aware encryption strategy},
% {\em adjustable query-based encryption}, and {\em onion encryption}.
% As part of \name's SQL-aware encryption strategy, we propose
% optimizations for existing cryptographic techniques, as well as a new
% cryptographic mechanism for private joins.  Our prototype of \name
% requires no changes to application or DBMS server code, and uses
% user-defined functions to perform cryptographic operations inside an
% existing DBMS engine, including both Postgres and MySQL\@.  Under a
% TPC-C workload, our prototype incurs a $\tput$ reduction in throughput
% compared to an unencrypted DBMS, making \name a practical option for
% privacy-sensitive data.

