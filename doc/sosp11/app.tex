% !TEX root = paper.tex

\section{Application Case Studies}
\label{s:apps}

In this section, we explain how \name can be used to secure three
existing multi-user web applications.  In our examples, annotations
are shown in bold.  For brevity, we show simplified schemas, ignoring
irrelevant fields and type specifiers.  Overall, we find that once the
programmer specifies the principals in the application's schema, and
the delegation rules for them using {\small \tt HAS\_ACCESS\_TO},
protecting additional sensitive fields just requires additional
{\small \tt ENC\_FOR} annotations.

{\bf phpBB} is a widely-used open source forum with a rich set of
access control settings. Users are organized in groups; both users and
groups have a variety of access permissions that the application
administrator can choose. We already showed how to secure private
messages between two users in phpBB in Fig.~\ref{fig:privmsg}.  A more
interesting case is securing access to posts, as shown in
Fig.~\ref{fig:posts}.  This example shows how to use predicates
(e.g., {\small \tt IF optionid=...}) to give conditional access to
principals, and also how one column ({\tt forumid}) can be used to
represent multiple principals (of different type) with different
privileges.  There are more ways to access to a post, but we do not
show them all here for brevity.

% phpBB example

\renewcommand{\FrameSep}{0.05in}
\begin{figure}[t!]
\begin{framed}
\footnotesize

\begin{tabbing}

{\bf PRINC TYPES physical\_user EXTERNAL;} \\
{\bf PRINC TYPES user, group, forum\_post, forum\_name;}\\

\\

x \= \kill

CREATE TABLE users ( userid int, username varchar(255),\\

\> {\bf PRINC username TYPE physical\_user HAS\_ACCESS\_TO } \\
\> {\bf ~ ~ ~ PRINC userid TYPE user});\\

\\

CREATE TABLE usergroup ( userid int, groupid int,\\

\> {\bf PRINC userid TYPE user HAS\_ACCESS\_TO } \\
\> {~ ~ ~ \bf PRINC groupid TYPE group});\\

\\

CREATE TABLE aclgroups ( groupid int, forumid int, optionid int, \\

\> {\bf PRINC groupid TYPE group HAS\_ACCESS\_TO }\\
\> {\bf ~ ~ ~ PRINC forumid TYPE forum\_post IF optionid=20}, \\
\> {\bf PRINC groupid TYPE group HAS\_ACCESS\_TO }\\
\> {\bf ~ ~ ~ PRINC forumid TYPE forum\_name IF optionid=14}); \\

\\
x \= \kill


CREATE TABLE posts ( postid int, forumid int,\\
\> post text {\bf ENC\_FOR PRINC forumid TYPE forum\_post});\\
\\
CREATE TABLE forum ( forumid int,\\
\> name varchar(255) {\bf ENC\_FOR PRINC forumid } \\
\> {\bf ~ ~ ~ TYPE forum\_name});

\end{tabbing}
\vspace{-0.15in}

\end{framed}

\caption{Annotated schema for securing access to posts in phpBB\@. A
  user has access to see the content of posts in a forum if any of the
  groups that the user is part of has such permissions, which is indicated
  by optionid 20 in the aclgroups table for the corresponding forumid
  and groupid. Similarly, optionid 14 enables users to see the forum's
  name.}

\label{fig:posts}
\end{figure}

% HOTCRP

{\bf HotCRP} is a popular conference review application.  A key policy
for HotCRP is that PC members cannot see who reviewed their own (or
conflicted) papers.  Fig.~\ref{fig:hotcrp} shows \name annotations for
HotCRP's schema to enforce this policy.  Today, HotCRP cannot prevent
a curious or careless PC chair from logging into the database server
and seeing who wrote each review for a paper that he is in conflict
with.  As a result, conferences often set up a second server to review
the chair's papers or use inconvenient out-of-band emails.  With
\name, a PC chair cannot learn who wrote each review for his paper,
even if he breaks into the application or database, since he does not
have the decryption key.\footnote{Fully implementing this policy would
  require setting up two PC chairs: a main chair, and a backup chair
  responsible for reviews of the main chair's papers.  HotCRP allows
  the PC chair to impersonate other PC members, so \name annotations
  would be used to prevent the main chair from gaining access to keys
  of reviewers assigned to his paper.} (We assume the PC chair is not
malicious, but might be careless, so he won't modify the application
to log the passwords of PC members to subvert the system.)


{\bf grad-apply} is a graduate admissions system used at a major
university.  We annotated its schema to allow an applicant's folder to be accessed only by the respective applicant and any faculty using: 
 {\tt \small PRINC reviewers.reviewer\_id TYPE reviewer} (meaning all reviewers) {\tt \small HAS\_ACCESS\_TO candidate\_id} in table candidates, and {\tt \small HAS\_ACCESS\_TO letter\_id} in table letters. The applicant can see all his folder except for letters of recommendation.  Overall, grad-apply has simple access control and therefore simple
annotations. 


%\begin{figure}
%\renewcommand{\FrameSep}{0.05in}
%\begin{framed}
%\footnotesize
%
%{\bf PRINC TYPES physical\_contact EXTERNAL;} \\
%{\bf PRINC TYPES contact, paper;}
%
%\begin{tabbing}
%
%x \= \kill
%
%CREATE TABLE ContactInfo ( contactId int, email varchar(120), \\
%
%\> {\bf PRINC email TYPE physical\_contact HAS\_ACCESS\_TO PRINC contactId TYPE contact});\\
%
%\\
%
%CREATE TABLE PaperConflict( paperId int, contactId int);\\
%\\
%
%{\bf NoConflict(paperId, contactId):} \\
%\> {\bf (SELECT count(*) FROM PaperConflict c WHERE c.paperId = paperId AND c.contactId = contactId)=0; } \\
%
%\\
%
%CREATE TABLE PaperReview ( paperId int, \\
%\> contactId int {\bf ENC\_FOR PRINC paperId TYPE paper}, \\
%\> commentsToPC text {\bf ENC\_FOR PRINC paperId TYPE paper},\\
%\> {\bf PRINC contactId TYPE contact HAS\_ACCESS\_TO PRINC paperId TYPE paper  IF NoConflict(paperId, contactId)} \\
%
%\end{tabbing}
%
%\end{framed}
%
%\caption{
%      Annotated schema for securing reviews in HotCRP\@. Reviews and the identity of reviewers providing the review will only be available to assigned reviewers who are not conflicted even if these reviewers are PC Chairs.}
%\label{fig:posts}
%\end{figure}

%HOTCRP any PC member can see reviews of a paper and who reviewed it other than conflicted PC members or PC Chairs

\begin{figure}
\renewcommand{\FrameSep}{0.05in}
\begin{framed}
\footnotesize

{\bf PRINC TYPES physical\_contact EXTERNAL;} \\
{\bf PRINC TYPES contact, review;}

\begin{tabbing}

x \= \kill


CREATE TABLE ContactInfo ( contactId int, email varchar(120), \\
\> {\bf PRINC email TYPE physical\_contact HAS\_ACCESS\_TO } \\
\> {\bf ~ ~ ~ PRINC contactId TYPE contact});\\

\\

CREATE TABLE PCMember ( contactId int ); \\

CREATE TABLE PaperConflict ( paperId int, contactId int );\\
\\

{\bf NoConflict (paperId, contactId):} ~ ~ {\tt /*} Define a SQL function {\tt */} \\
\> {\bf (SELECT count(*) FROM PaperConflict c  WHERE }\\
\> {~ ~ ~ \bf c.paperId = paperId AND c.contactId = contactId) = 0;} \\

x \= commentsToPC \= text \= {\bf ENC\_FOR PRINC paperId TYPE review} \= \kill
\\
CREATE TABLE PaperReview ( paperId int, \\
\> reviewerId \> int \> {\bf ENC\_FOR PRINC paperId TYPE review}, \\
\> commentsToPC \>text \> {\bf ENC\_FOR PRINC paperId TYPE review},\\


\> {\bf PRINC PCMember.contactId TYPE contact }\\
\> {\bf ~ ~ ~ HAS\_ACCESS\_TO PRINC paperId TYPE review IF } \\
\> {\bf ~ ~  ~ ~ ~ ~ ~ ~ ~ ~ ~ ~ ~ ~ ~ ~ ~ NoConflict(paperId, contactId)});

\end{tabbing}
\vspace{-0.15in}
\end{framed}
\caption{Annotated schema for securing reviews in HotCRP\@. Reviews and
the identity of reviewers providing the review will only be available to
PC members (table PCMember includes PC Chairs) who are not conflicted,
and PC Chairs cannot override this restriction.}

\label{fig:hotcrp}
\end{figure}



