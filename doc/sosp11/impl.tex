% !TEX root = paper.tex

\section{Implementation}
\label{s:impl}

The \name{} proxy consists of a C++ and a Lua module.  The C++ module
consists of a query parser; a query encryptor and rewriter, which
encrypts fields or calls UDFs; and a result decryption module.  To
allow applications to transparently use \name, we used MySQL Proxy and
implemented a Lua module that passes queries and results through our
C++ module.  We implemented our new cryptographic protocols using
NTL~\cite{shoup:ntl}.  \name{} is $\sim 8500$ non-empty lines of code.

\name is portable and supports both Postgres 9.0 and MySQL 5.1;
initially implemented in Postgres, porting \name{} to MySQL required
changing only 86 lines of code, mostly in the code for connecting to the
MySQL server and declaring UDFs.  As mentioned earlier, \name{} does not
change the DBMS; we implement all server-side functionality with
UDFs and server-side tables.  This modular change was possible because
the DBMS lies in between two aspects of query processing that \name{}
must modify: query planning and execution is between query parsing
and low-level operations on data items.
\name should run on top of any SQL DBMS that supports UDFs.

% by rewriting queries in the proxy and replacing individual operations
% through UDFs.


