% !TEX root = paper.tex

\begin{abstract}

  Online applications are vulnerable to the theft of sensitive
  information because adversaries can exploit software bugs to gain
  access to private data, and because curious or malicious
  administrators may capture and leak data.  \name is a system that
  provides practical and provable confidentiality in the face of these
  attacks for applications backed by SQL databases.  It works by fully
  {\em executing SQL queries over encrypted data} using a collection of
  efficient SQL-aware encryption schemes.  \name also {\em
    chains encryption keys to user passwords}, so that a data item can
  be decrypted only using the password of one of the users with access
  to that data.  As a result, even if the adversary compromises all servers,
  he or she cannot decrypt the data of any user that is not logged in.
  Our evaluation shows that \name has low overhead: on a real application, phpBB, \name{} reduces throughput by 
  13\%, and on the TPC-C
  benchmark by 27\% compared to regular
  Postgres.  Chaining encryption keys to user passwords requires
  11--13 unique schema annotations for database schemas of three
  multi-user web applications to capture their policies.

  % Encrypting each user's data with different keys requires XXX lines of
  % annotations in phpBB (a popular web bulletin board) and XXX lines of
  % annotations in the HotCRP conference review system.

%   Importantly, \name does not change the innards of existing DBMSs:
%   we realized the implementation of \name using client-side query
%   rewriting/encrypting, user-defined functions, and server-side tables
%   for public key information. As such, \name is portable; porting
%   \name to MySQL required changing 86 lines of code, mostly at the
%   connectivity layer.

\end{abstract}
