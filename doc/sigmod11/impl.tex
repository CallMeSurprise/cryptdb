\section{Implementation}
\label{s:impl}

We implemented CryptDB in C++ on top of Postgres 9.0, and later ported it
to MySQL, as reported in \S\ref{ss:mysqlport}.  We used the NTL library
for doing number theory on large numbers~\cite{shoup:ntl} to implement
some of our cryptographic protocols. CryptDB is $~4700$ lines of code,
not counting empty lines, standard libraries, the NTL library, or
evaluation code.

As mentioned earlier, CryptDB does not change the innards of a DBMS\@.
We managed to implement all the server-side functionality with UDFs and
server-side tables.  The insight into why such modular change was possible is that
the DBMS in fact lies in between two aspects of query processing that
CryptDB must modify.  Specifically, query planning and execution is
between query parsing and low-level operations on data items.
Therefore, CryptDB can
run on top of any DBMS that is SQL-compatible and supports UDFs, by
rewriting queries in the frontend and replacing individual operations
through UDFs.

